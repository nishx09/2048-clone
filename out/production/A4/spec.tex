\documentclass[12pt]{article}

\usepackage{graphicx}
\usepackage{paralist}
\usepackage{amsfonts}
\usepackage{amsmath}
\usepackage{hhline}
\usepackage{booktabs}
\usepackage{multirow}
\usepackage{multicol}
\usepackage{url}

\oddsidemargin -10mm
\evensidemargin -10mm
\textwidth 160mm
\textheight 200mm
\renewcommand\baselinestretch{1.0}

\pagestyle {plain}
\pagenumbering{arabic}

\newcounter{stepnum}

%% Comments

\usepackage{color}

\newif\ifcomments\commentstrue

\ifcomments
\newcommand{\authornote}[3]{\textcolor{#1}{[#3 ---#2]}}
\newcommand{\todo}[1]{\textcolor{red}{[TODO: #1]}}
\else
\newcommand{\authornote}[3]{}
\newcommand{\todo}[1]{}
\fi

\newcommand{\wss}[1]{\authornote{blue}{SS}{#1}}

\title{Assignment 4, Design Specification}
\author{SFWRENG 2AA4}

\begin{document}

\maketitle
This Module Interface Specification (MIS) document contains modules, types and
methods for implementing the game \textit{Two Dots}. At the start of each game, the user can choose
either to eliminate the targetted amount of dots within a specified number of moves or within a time limit.
The user must connect at least two dots of the same color to make a move and remove the dots. The dots can 
be connected horizontally or vertically, but not diagonally,
dots directly above those eliminated dots will then drop down to fill in the gap. Throughout this specification 
document, each dot will be referred as to a colored dot and a cell refers to the space occupied by a dot. In addition
Row number increases when moving from top to bottom and column number increases when moving from left to right. The 
game can be launched and play by typing \texttt{make demo} in terminal.

\begin{center}
  \includegraphics[width=0.7\textwidth]{naming.png}

  The above board visualization is from https://www.gamesxl.com/think/two-dots
\end{center}

\newpage

\section{Overview of the design}

This design applies Module View Specification (MVC) design pattern, Strategy design pattern and Singleton design pattern. The MVC components
are \textit{GameController} (controller module), \textit{BoardT} (model module), and \textit{UserInterface} (view module). For the Strategy design pattern,
the \textit{EndCondition} (interface) captures the abstraction and the derived classes: \textit{EndByTime} and \textit{EndByMoves} encapsulate the implementation details.
Singleton pattern is specified and implemented for \textit{GameController} and \textit{UserInterface}

\bigskip

\noindent An UML diagram is provided below for visualizing the structure of this software architecture

\includegraphics[width=0.9\textwidth]{UML.png}

\medskip
The MVC design pattern is specified and implemented in the following way: the module \textit{BoardT}
stores the state of the game board and the status of the game. A view module \textit{UserInterface} can display
the state of the game board and game using a text-based graphics. The controller \textit{GameController}
is responsible for handling input actions. 

\medskip

The implementation of the Strategy pattern enables the game ending condition to be changeable during
runtime. The user can choose the game to end after
a time limit \textit{EndByTime} or within a specified amount fo moves (\textit{EndByMoves}).

\medskip

For \textit{GameController} and \textit{UserInterface}, use the getInstance() method to obtain the abstract object.

\newpage

\subsection*{Likely Changes my design considers:}

\begin{itemize}
  \item Data structure used for storing the game board
  \item The visual representation of the game such as UI layout. 
  \item Change in peripheral devices for taking user input. 
  \item Change in game ending conditions to adjust the difficulty of the game.
\end{itemize}

\newpage

\section* {Cell Type Module}

\subsection*{Module}

CellType

\subsection* {Uses}

N/A

\subsection* {Syntax}

\subsubsection* {Exported Constants}

None

\subsubsection* {Exported Types}

CellType = \{R, G, B, P, O, E\}

\medskip

\noindent \textit{//R stands for Red, G for Green, B for Blue, \\
                    P for Purple, O for Orange, E for Empty (Placeholder)}

\subsubsection* {Exported Access Programs}

None

\subsection* {Semantics}

\subsubsection* {State Variables}

None

\subsubsection* {State Invariant}

None

\newpage

\section* {EndCondition Interface Module}

\subsection* {Interface Module}

EndCondition

\subsection*{Uses}

None

\subsection* {Syntax}

\subsection*{Exported Constants}

None

\subsection*{Exported Types}

None

\subsubsection* {Exported Access Programs}

\begin{tabular}{| l | l | l | p{6cm} |}
\hline
\textbf{Routine name} & \textbf{In} & \textbf{Out} & \textbf{Exceptions}\\
\hline
gameStatus & $\mathbb{N}$, $\mathbb{N}$, $String$ & $\mathbb{B}$ & \\
\hline
getMessage &  & $String$ & \\
\hline
\end{tabular}

\newpage

\section* {EndByMoves Module}

\subsection* {Template Module implements EndCondition Interface}

EndByMoves

\subsection*{Uses}

EndCondition

\subsection* {Syntax}

\subsection*{Exported Constants}

None

\subsection*{Exported Types}

None

\subsubsection* {Exported Access Programs}

\begin{tabular}{| l | l | l | p{6cm} |}
\hline
\textbf{Routine name} & \textbf{In} & \textbf{Out} & \textbf{Exceptions}\\
\hline
EndByMoves & ~ & EndByMoves & \\
\hline
\end{tabular}

\subsection* {Semantics}

\subsection*{State Variables}

maxMoves: $\mathbb{N}$ \\
redTarget: $\mathbb{N}$ \\
orangeTarget : $\mathbb{N}$ 

\subsection*{State Invariant}

None

\subsection*{Assumptions}

None

\subsubsection* {Access Routine Semantics}

new EndByMoves():

\begin{itemize}
  \item output: out $:=$ self
  \item transition: maxMoves, redTarget, orangeTarget $:=$ 10, 7, 7
\end{itemize}

\noindent gameStatus(num, moves, color):

\begin{itemize}
  \item transition $:$
        redTarget, orangeTarget $:=$ (color = CellType.R $\Rightarrow$ redTarget - num), (color = CellType.O $\Rightarrow$ orangeTarget - num)
  \item output:
  \medskip
  \begin{tabular}{| l | l |}
  \hline
  ~ & $out$ $:=$ \\
  \hline
  redTarget $<=$ 0 $wedge$ orangeTarget $<=$ 0 & false \\
  \hline
  (maxMoves - moves) $<=$ 0 & false \\
  \hline
  redTarget $>$ 0 $wedge$ orangeTarget $>$ 0 & true \\
  \hline
  (maxMoves - moves) $>$ 0 & true \\
  \hline
  \end{tabular}
\end{itemize}

\noindent getString():
\begin{itemize}
  \item output:$out$ $:=$ A string containing information about the number of red dots
                          , orange dots, and moves left
\end{itemize}

\newpage

\section* {EndByTime Module}

\subsection* {Template Module implements EndCondition Interface}

EndByTime

\subsection*{Uses}

EndCondition, Runnable

\subsection* {Syntax}

\subsection*{Exported Constants}

None

\subsection*{Exported Types}

None

\subsubsection* {Exported Access Programs}

\begin{tabular}{| l | l | l | p{6cm} |}
\hline
\textbf{Routine name} & \textbf{In} & \textbf{Out} & \textbf{Exceptions}\\
\hline
EndByTime & ~ & EndByTime    & \\
\hline
getTime & ~ & $\mathbb{N}$ & \\
\hline
setTime & $\mathbb{N}$ & ~ & \\
\hline
startCountDown & ~ & ~ & \\
\hline
\end{tabular}

\subsection* {Semantics}

\subsection*{Environment Variables}

t: Thread

\subsection*{State Variables}

timeLeft: $\mathbb{N}$ \\
redTarget: $\mathbb{N}$ \\
orangeTarget: $\mathbb{N}$ \\

\subsection*{State Invariant}

None

\subsection*{Assumptions}

None

\subsubsection* {Access Routine Semantics}

new EndByTime():

\begin{itemize}
  \item output: out $:= \mathit{self}$
  \item transition: redTarget, orangeTarget, t $:=$ 5, 5, new Thread(Run(this)) \\ \textit{// initiates the timer in thread t}
\end{itemize}

\noindent gameStatus(num, moves, color):

\begin{itemize}
  \item transition $:$
        redTarget, orangeTarget $:=$ (color = CellType.R $\Rightarrow$ redTarget - num), (color = CellType.O $\Rightarrow$ orangeTarget - num)
  \item output:
  \medskip
  \begin{tabular}{| l | l |}
  \hline
  ~ & $out$ $:=$ \\
  \hline
  redTarget $<=$ 0 $wedge$ orangeTarget $<=$ 0 & false \\
  \hline
  timeLeft $<=$ 0 & false \\
  \hline
  redTarget $>$ 0 $wedge$ orangeTarget $>$ 0 & true \\
  \hline
  timeLeft $>$ 0 & true \\
  \hline
  \end{tabular}
\end{itemize}

\noindent getString():
\begin{itemize}
  \item output: $out :=$ a String that reveal the amount of time , red dots, and orange dots left until game terminates and
\end{itemize}

\noindent getTime():
\begin{itemize}
  \item output: $out :=$ timeLeft
\end{itemize}

\noindent setTime(time):
\begin{itemize}
  \item transition: $timeLeft :=$ time
\end{itemize}

\noindent startCountDown():
\begin{itemize}
  \item transition: t.start() \quad \textit{// Starts the count down clock in the thread}
\end{itemize}


\newpage

\section* {Board ADT Module}

\subsection*{Template Module inherits EndCondition}

BoardT (with EndCondition)

\subsection* {Uses}

CellType, EndCondition

\subsection* {Syntax}

\subsubsection* {Exported Types}

None

\subsubsection* {Exported Constant}

Size = 8 \quad // Size of the board 8 x 8

\subsubsection* {Exported Access Programs}

\begin{tabular}{| l | l | l | l |}
\hline
\textbf{Routine name} & \textbf{In} & \textbf{Out} & \textbf{Exceptions}\\
\hline
BoardT & ~ & BoardT & \\
\hline
BoardT & seq of (seq of CellType) & BoardT & \\
\hline
getStatus & ~ & $\mathbb{B}$ & \\
\hline
getCell & $\mathbb{N}$, $\mathbb{N}$ & CellType & IndexOutOfBoundsException\\
\hline
getMessage & ~ & String & \\
\hline
setCondition & EndCondition & ~ & \\
\hline
remove & seq of String & ~ & IllegalArgumentException\\
\hline
replaceEliminated  &  seq of String  &  ~     & \\
\hline
\end{tabular}

\subsection* {Semantics}

\subsubsection* {State Variables}

board: sequence [Size, Size] of CellType \\
status: $\mathbb{B}$ \\
condition: EndCondition \\
moves: $\mathbb{N}$

\subsubsection* {State Invariant}

None

\subsubsection* {Assumptions}

\begin{itemize}
  \item The constructor BoardT is called for each object instance before any other access routine 
  is called for that object. 
  \item Assume there is a random function that generates a random value beteern 0 and 1.
  \item There will always be two adjacent color dots to connect.
  \item remove method is called before replaceEliminated method.
\end{itemize}

\subsubsection* {Design decision}

The coordinates of the board is stored in a 2D sequence. The coordinates used for retriving the cell is different
from the coordinate entered by the user. When the user refers to row 0 and column 0 cell 
on their screen, that cell is actually stored in board[0][7]. Thus, board[0][0] is located at bottom left corner of the UI 
and board[7][7] is located at top right corner of the UI.

\begin{center}
  \includegraphics[width=0.8\textwidth]{coord.png} \\
  In the above illustration, cell at (0,0) is stored in board[0][7].
\end{center}

\begin{itemize}
  \item The main reason I store the board in this way is for the ease of dots removal process. It is easier to perform
        the shift dot operation if those dots are in the same row. Since each row of the board is represented as a
        column on the UI. A shift left operation in a row will look like a drop down on the UI.
\end{itemize}

\subsubsection* {Access Routine Semantics}

BoardT():
\begin{itemize}
\item transition: \\
      board $:=$ 
      $\langle \begin{array}{c}
      \langle \mbox{randomCellType()}_0, ... ... ,\mbox{randomCellType()}_7 \rangle\\
      \langle \mbox{randomCellType()}_0, ... ... ,\mbox{randomCellType()}_7 \rangle\\
      \langle \mbox{randomCellType()}_0, ... ... ,\mbox{randomCellType()}_7 \rangle\\
      \langle \mbox{randomCellType()}_0, ... ... ,\mbox{randomCellType()}_7 \rangle\\
      \langle \mbox{randomCellType()}_0, ... ... ,\mbox{randomCellType()}_7 \rangle\\
      \langle \mbox{randomCellType()}_0, ... ... ,\mbox{randomCellType()}_7 \rangle\\
      \langle \mbox{randomCellType()}_0, ... ... ,\mbox{randomCellType()}_7 \rangle\\
      \langle \mbox{randomCellType()}_0, ... ... ,\mbox{randomCellType()}_7 \rangle\\
      \end{array} \rangle$ \\ 
      status, moves $=$ true, 0
\item output: $out := \mathit{self}$
\item exception: None
\end{itemize}

\noindent BoardT($b$):
\begin{itemize}
\item transition: board, status, moves $=$ $b$, true, 0
\item output: $out := \mathit{self}$
\item exception: None
\end{itemize}

\noindent getStatus():
\begin{itemize}
\item transition: none
\item output: $out :=$ status
\item exception: None
\end{itemize}

\noindent getMessage():
\begin{itemize}
\item transition: none
\item output: $out :=$ (condtion.message(moves))
\item exception: None
\end{itemize}

\noindent getCell($x, y$):
\begin{itemize}
\item output: $out :=$ board[$y$][7 - $x$]
\item exception: $exc :=$ ($\neg$ validateCell($x, y) \Rightarrow$ IndexOutOfBoundsException)
\end{itemize}

\noindent setCondition($c$):
\begin{itemize}
\item transition: condition $:=$ $c$
\item output: None
\item exception: None
\end{itemize}

\noindent remove($input$):
\begin{itemize}
  \item transition: board $:=$ markCell($input$) $\Rightarrow$ ($\forall i : \mathbb{N}$ $|$ $i < size$ $\wedge$ board[$i$].removeAll($CellType.E$))
  \item output: None
  \item exception: $exc :=$ (($\neg$ validateConnectivity($input) \Rightarrow$ IllegalArgumentExceptio) $|$ ()$|input|$ = 1 $\Rightarrow$ IllegalArgumentException))
\end{itemize}

\noindent replaceEliminated($input$):
\begin{itemize}
\item transition: board $:=$\\
  ($|board[i]|$ $<$ size $\Rightarrow$ ($\forall$ $i : \mathbb{N}$ $|$ $|input| <= i < size$ . board[$i$] $||$ <randomCellType()>)) \\
  \medskip
  \textit{// Add randomized colored cell to a row if that row is missing some cells}
\end{itemize}

\subsubsection* {Local Functions}

validateCell: $\mathbb{N} \times \mathbb{N} \rightarrow \mathbb{B}$ 

\medskip

\noindent validateCell($x, y$) $\equiv$ $x <$ Size $\wedge$ $y <$ Size $\wedge$ $x >=$ 0 $\wedge$ $y >=$ 0 

\vspace{1.5\baselineskip}

\noindent validateConnectivity: sequence of String $\rightarrow \mathbb{B}$

\medskip

\begin{center}
\begin{tabular}{|c|c|p{5cm}} 
\hline
validateConnectivity(input) $\equiv$ $\forall i : \mathbb{N}| i > 0 \wedge i < 8$ . & $getCell(x,y) == getCell(x_1,y_1)$\\
$x = input[i][0], y = input[i][1]$ & $(x + 1 == x1 \vee x - 1 == x1) \wedge (y == y1)$ \\ 
$x_1 = input[i][0], y_1 = input[i-1][1]$ . & $(y + 1 == y1 \vee y - 1 == y1) \wedge (x == x1)$  \\ 
\hline
\end{tabular}
\end{center}

\textit{// Checks the current cell with the previous one. It is valid if they both have the same color and adjacent meaning 
            either horizontally or vertically connected}

\bigskip
\bigskip

\noindent markCell: sequence of String \\

\noindent markCell: $\forall$ $i : \mathbb{N}$ $|$ $i < |input|$ $\wedge$ board[$input[i].charAt[0]][input[i].charAt[1]$] = $CellType.E$

\textit{// mark each coordinate in the input as CellType.E (empty cell)}

\newpage

\section* {UserInterface Module}

\subsection* {UserInterface Module}

\subsection* {Uses}

None

\subsection* {Syntax}

\subsubsection* {Exported Types}

None

\subsubsection* {Exported Constants}

None

\subsubsection* {Exported Access Programs}

\begin{tabular}{| l | l | l | p{6cm} |}
\hline
\textbf{Routine name} & \textbf{In} & \textbf{Out} & \textbf{Exceptions}\\
\hline
getInstance & ~ & UserInterface &  \\
\hline
printBoard & BoardT & ~ & \\
\hline
printWelcomeMessage & ~ & ~ & \\
\hline
printGameModePrompt & ~ & ~ & \\
\hline
printCoordPrompt & ~ & ~ & \\
\hline
printCondition & String & ~ & \\
\hline
printEndingMessage & ~ & ~ & \\
\hline
\end{tabular}

\subsection* {Semantics}

\subsection*{Environment Variables}

window: A portion of computer screen to display the game and messages

\subsubsection* {State Variables}

visual: UserInterface

\subsubsection* {State Invariant}

None

\subsubsection* {Assumptions}

\begin{itemize}
\item The UserInterface constructor is called for each object instance before any
other access routine is called for that object.  The constructor can only be
called once.
\end{itemize}

\subsubsection* {Access Routine Semantics}

\noindent getInstance():
\begin{itemize}
  \item transition: visual $:=$ (visual = null $\Rightarrow$ new UserInterface())
  \item output: \textit{self}
  \item exception: None
\end{itemize}

\noindent printWelcomeMessage():
\begin{itemize}
\item transition: window $:=$ Displays a welcome message when user first enter the game.
\end{itemize}

\noindent printBoard($board$):
\begin{itemize}
\item transition: window $:=$ Draws the game board onto the screen. Each cell of the board
                  is accessed using the $getCell$ method from $BoardT$. The board[x][y] is displayed 
                  in a way such that x is increasing from the left of the screen to the right,
                  and y value is increasing from the bottom to the top of the screen. For example,
                  board[0][0] is displayed at the bottom left corner and board[7][7] is displayed 
                  at top-right corner. 
\end{itemize}

\noindent printGameModePrompt():
\begin{itemize}
\item transition: window $:=$ Window appends a prompt message asking the user to select a
                              game mode he/she wants to play.
\end{itemize}

\noindent printCoordPrompt():
\begin{itemize}
\item transition: window $:=$ Appends a prompt message asking the user to enter a sequence
                              of coordinates where the dots will be eliminated.
\end{itemize}

\noindent printCondition($message$):
\begin{itemize}
\item transition: Appends the $message$ onto the screen, the $message$ shows the objective of current game. Also, the amount of moves
                  of number of specific dots remanin to complete the game.
\end{itemize}

\noindent printEndingMessage():
\begin{itemize}
\item transition: Prints a ending message after the user exit the game (entered ``e'').
\end{itemize}

\subsubsection*{Local Function:}

UserInterface: void $\rightarrow$ UserInterface \\
UserInterface() $\equiv$ new UserInterface()

\newpage

\section* {GameController Module}

\subsection* {GameController Module}

\subsection* {Uses}

BoardT, UserInterface

\subsection* {Syntax}

\subsubsection* {Exported Types}

None

\subsubsection* {Exported Constants}

None

\subsubsection* {Exported Access Programs}

\begin{tabular}{| l | l | l | p{4.7cm} |}
\hline
\textbf{Routine name} & \textbf{In} & \textbf{Out} & \textbf{Exceptions}\\
\hline
getInstance & BoardT, UserInterface & GameController & ~ \\
\hline
initializeGame & ~ & ~ & ~\\
\hline
setGameMode & String & ~ & ~ \\
\hline
setElimination & Sequence of String & ~ & ~ \\
\hline
getStatus& ~ & $\mathbb{B}$ & ~ \\
\hline
readGameModeInput& ~ & String & IllegalArgumentException \\
\hline
displayWelcomeMessage& ~ & ~ & ~ \\
\hline
displayBoard& ~ & ~ & ~ \\
\hline
displayCondition& ~ & ~ & ~ \\
\hline
displayEnding& ~ & ~ & ~ \\
\hline
displayGameModePrompt& ~ & ~ & ~ \\
\hline
displayCoordPrompt& ~ & ~ & ~ \\
\hline
runGame & ~ & ~ & ~ \\
\hline
\end{tabular}

\subsection* {Semantics}

\subsection*{Environment Variables}

keyboard: Scanner(System.in) \qquad \textit{// reading inputs from keyboard}

\subsubsection* {State Variables}

model: BoardT \\
view: UserInterface \\
controller: GameController

\subsubsection* {State Invariant}

None

\subsubsection* {Assumptions}

\begin{itemize}
  \item The GameController constructor is called for each object instance before any
  other access routine is called for that object.  The constructor can only be
  called once.
  \item Assume that model and view instances are already initialized before calling GameController
        constructor
\end{itemize}

\subsubsection* {Access Routine Semantics}

getInstance($m$, $v$):
\begin{itemize}
  \item transition: controller $:=$ (controller = null $\Rightarrow$ new GameController ($m, v$))
  \item output: \textit{self}
  \item exception: None
\end{itemize}

\noindent initializeGame():
\begin{itemize}
  \item transition: model $:=$ $new BoardT()$
  \item output: None
  \item exception: None
\end{itemize}

\noindent setGameMode($input$):
\begin{itemize}
  \item transition:
        \begin{tabular}{| l | l |}
          \hline
          ~ & $out :=$ \\
          \hline
          $input$ = `m' & model.setEndCondition(new EndByMoves()) \\
          \hline
          $input$ = `t' & cond = new EndByTime() $\Rightarrow$ model.setEndCondition(cond) $\wedge$ \\
                           &                                     cond.startCountDown()  \\
          \hline
        \end{tabular}
  \item output: None
  \item exception: None
\end{itemize}

\noindent setElimination($s$):
\begin{itemize}
  \item transition: model.board $:=$ (model.eliminate($s$))
  \item output: None
  \item exception: None
\end{itemize}

\noindent replaceElimination($s$):
\begin{itemize}
  \item transition: model.board $:=$ (model.replaceEliminated($s$))
  \item output: None
  \item exception: None
\end{itemize}

\noindent getStatus():
\begin{itemize}
  \item transition: None
  \item output: $out$ $:=$ (model.getStatus())
  \item exception: None
\end{itemize}

\noindent readGameModeInput():
\begin{itemize}
  \item output: $input$ : String, entered from the keyboard by the User
  \item exception: $exc$ $:=$ (input $\neq$ ``m'' $\wedge$ input $\neq$ ``t'' $\wedge$ input $\neq$ ``e'' $\Rightarrow$ IllegalArgumentException)\\
                              \textit{// ``m'' for maximum amount of moves, ``t'' for time limit, ``e'' to exit the game}
\end{itemize}

\noindent readCoord():
\begin{itemize}
  \item output: $out :=$ $input$ \textit{// a String, entered from the keyboard by the User} 
\end{itemize}

\noindent displayWelcomeMessage():
\begin{itemize}
  \item transition: view $:=$ view.printWelcomeMessage()
\end{itemize}

\noindent displayBoard():
\begin{itemize}
  \item transition: view $:=$ view.printBoard()
\end{itemize}

\noindent displayCondition():
\begin{itemize}
  \item transition: view $:=$ view.printCondition(model.getMessage())
\end{itemize}

\noindent displayEnding():
\begin{itemize}
  \item transition: view $:=$ view.printEndingMessage()
\end{itemize}

\noindent displayGameModePrompt():
\begin{itemize}
  \item transition: view $:=$ view.printGameModePrompt()
\end{itemize}

\noindent displayCoordPrompt():
\begin{itemize}
  \item transition: view $:=$ view.printCoordPrompt()
\end{itemize}

\noindent runGame():
\begin{itemize}
  \item transition: operational method for running the game. The game will start with a welcome message, next
                    asking the user to select a game objective, then display the board and let the user to play the game. 
                    Eventually, when the game ends, prompt a message asking the user to play another round or exit the game.
  \item output: None
\end{itemize}

\subsubsection*{Local Function:}

GameController: BoardT $\times$ UserInterface $\rightarrow$ GameController \\
GameController($model, view$) $\equiv$ new GameController($model, view$)

\newpage

\section*{Critique of Design}

\begin{itemize}
  \item I choose to specify BoardT module as ADT over abstract object, because It is more convenient to create a new instance of the board after the 
        user choose to restart a game.
  \item The controller and view modules are specified as a single abstract object because these modules are shared resources and only one instance is required
        to control the action during runtime. Thus, any conflicting or unexpected state changes can be avoided.
  \item The $getCell()$ method in $BoardT$ module is not essential. I added this method 
        is mainly for the high usability for the view module to display the status of the game.
  \item EndCondition interface provides some to generality to this design when comes to solving the problem of switching objective.
        I choose it to be an interface instead of a generic module because, in Java, you can only implement multiple interfaces but only one inheritance. Also, it 
        is easier to apply the strategy design pattern.
  \item The $gameStatus$ method in $EndByMoves$ and $EndByTime$ violates the principle of minimality. I design this module in this way is to ensure
        there is no delay or friction with the model module since the method updates the status of the game after each move made by the user during runtime.
  \item The two constructors in BoardT improve the flexibilty of the module. The user can choose to play with a board 
        initialized with randomly generated dots or a board that is customized or pre-defined. Also, from a testing perspective, methods can be 
        easily tested if the board is pre-defined compared to a randomly generated board.
  \item The test cases are designed to validate the correctness of the program based on the requirement and reveal errors
        or unusual behavior during program execution, every access routine has at least one test case. One exception is
        made for $replaceEliminated$ method in BoardT because the $replaceEliminated$ method adds a random dot to the board, 
        there are no efficient ways to test the correctness of adding a randomly generated cell.
  \item In $BoardT$, the result of $remove$ and $replaceEliminated$ method is tested using $getMessage$ method. These methods do
        not output anything and $getMessage$ method stores the information of the scoreboard, it updates after each successful dot removal.
  \item Did not build any test cases for testing the controller module since the implementation of the controller's access methods uses 
        methods from the model and view. The test cases for the model are in $TestBoardT.java$
  \item Using MVC also makes my design maintainable and reduces risk when making changes. MVC decomposes into three 
        component based on the separation of concerns where the model component encapsulates the internal 
        data and status of the game, the view displays the state of the game, and the controller handles
        the input actions to execute related actions to respond to events.
  \item My design achieves high cohesion and low coupling by applying MVC. MVC keeps high cohesion since
        it groups related functionalities within each module. The deisgn is also low coupling because
        the modules (model, view, controller) are mostly independent of each other. So a change in one of the
        modules does not heavily impact the other.
  \item A strategy design pattern provides convenience and flexibility when comes to design for change. It is easier to switch
        between different algorithms during runtime through polymorphism with the Strategy pattern. In addition, it increases
        maintainability and readability in a way that the concerns are separated into classes instead of using
        conditional statements to switch strategy in runtime.
  \item I found using the Singleton design pattern is better than using static methods for abstract objects. During the developing process, using 
        static methods and variables usually cause some warnings about the method or variables to need to be accessed statically. Using singleton
        pattern eliminates all these problems, making a smoother development.
  
\end{itemize}

\section*{Answers to Questions:}

Q1: Please compare and contrast the proxy, strategy and adapter design
  patterns.  The UML diagrams for each look similar, but the purpose of the
  patterns are different.  What do the patterns have in common?  How are they
  different?  In which situations would you use each pattern?

\medskip
\medskip

\noindent \textbf{Similiarities:}

\begin{itemize}
  \item All the design pattern address issues of change in software
  \item Each pattern is for solving a particular kind of problem
  \item Implementation details are encapsulated in the derived classes and derived classes implements the interface
\end{itemize}

\medskip
\noindent \textbf{Differences:}

\noindent Proxy Design Pattern
\begin{itemize}
  \item Proxy design pattern is a structural design pattern. Proxy pattern is a placehold for another object to control access
        to it by adding a wrapper and delegation to protect the main object. 
  \item Used when direct access into another component is inefficient and unsafe.
\end{itemize}

\noindent Adapter Design Pattern
\begin{itemize}
  \item Adapter pattern is a structural design pattern. Adapter pattern allows the interface of an existing class 
        to be used as another interface.
  \item Used when the interface from an existing library does not match the requirement. Often used to make existing classes 
        compatible with each other without modifying the implementation of the existing classes.
\end{itemize}

\noindent Strategy Design Pattern 
\begin{itemize}
  \item Classifies into behaviour design pattern category and divides related logic into three components by separation of concerns.
        Uses a controller module to interconnect the components.
  \item Commonly used when a developer wish to create an interchangeable family of algorithm and switch algorithm during runtime.
\end{itemize}

\noindent Q2: Draw a control flow graph for bubble sort algorithm

\begin{center}
  \includegraphics[width=0.3\textwidth]{Control_Flow_Graph.png} \\
  The control flow graph is constructed using https://app.diagrams.net/
\end{center}

\end {document}